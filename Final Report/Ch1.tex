\selectlanguage{australian}%

\chapter{Introduction}

\acresetall

The ``cocktail party problem'' was first posed in 1953 by \cite{Cherry1953},
where the human ability, or often difficulty, to hear speech in the
presence of multiple speakers was noted. After analysis into the complexity
of the problem, it is amazing that humans have the ability to hear
over one-another at all!

The cocktail party problem refers to the problem of recognising speech
in the presence of babble. A number of speakers are present, and each
can be distinguished individually. This has been noted as an extremely
difficult task in speech analysis and enhancement. With the rise of
modern technology and the desire to incorporate alternative human-machine
interfaces, the motivation to improve \ac{ASR} systems has increased.
Additionally, the problem still exists of aiding human understanding
in such situations, e.g. hearing aid systems or telecommunications
systems.

One method of subspace analysis that has shown promising results is
that of \ac{NMF}. This is a relatively new method of decomposition
proposed by \cite{Lee1999}, with applications to spectral analysis
due to the non-negativity constraints of \ac{NMF} and the non-negative
nature of spectral magnitude data. The components of a desired signal
and be learned and extracted from a signal. Babble filtering systems
are required to be trained to recognise the individual speaker, which
is often a difficult process and a practical limitation in these systems.

A number of different challenges have been held with the motivation
of improving the performance of \ac{ASR} systems under difficult
noise conditions \cite{Cooke2010,Barker2013,Vincent2013}. Entries
into such competitions can be broadly categorised into two categories,
those that perform recognition themselves, and those that clean the
signal and supply a cleaned signal to a standardised recogniser. Algorithms
that fall into the latter category have a possible additional application:
improving intelligibility for human listeners. It is these algorithms
that are of interest in this thesis.


\section{Research Questions}

\newcommand{\RQone}{``Are good enhancement algorithms effective for both human listeners and machine listeners? Can a generic and practical speech enhancement algorithm find application in signal enhancement and \ac{ASR}?''}
\newcommand{\RQthree}{``Can the results be improved by modifying algorithms to concentrate the focus on recognition of the desired speakers voice?''}
\newcommand{\RQtwo}{``Can the practicality of existing algorithms be improved to allow applications in end-user hardware?''}

The aim of this thesis is to address the following research questions:
\begin{enumerate}
\item \label{enu:ResQ1} \RQone{}
\item \label{enu:ResQ2}\RQtwo{}
\item \label{enu:ResQ3}\RQthree{}
\end{enumerate}

\section{Scope}

The scope of this thesis is to develop a general and practical speech
enhancement algorithm with applications in a number of areas, including
enhancement for human listeners and for \ac{ASR} systems. Such an
algorithm should be efficient, have low requirements, and operate
on monaural recordings.

The scope does not encompass specific applications, and thus does
not consider special requirements beyond normal hearing for hearing
aid and cochlear implant listeners. It is assumed that the performance
increases for normal hearing listeners will be applicable to impaired
hearing listeners. Further studies would be required to ensure this
is the case.\selectlanguage{english}%

