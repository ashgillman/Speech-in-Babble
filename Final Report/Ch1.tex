
\chapter{Introduction}

\acresetall

The ``cocktail party problem'' was first posed in 1953 by \citet{Cherry1953},
where the human ability, or often difficulty, to hear speech in the
presence of multiple speakers was noted. After analysis into the complexity
of the problem, the human ability to hear voices presented simultaneously
is noted as incredible.

The cocktail party problem refers to the problem of recognising speech
in the presence of babble. A number of speakers are present, and each
can be distinguished individually. This has been noted as an extremely
difficult task in speech analysis and enhancement. With the rise of
modern technology and the desire to incorporate alternative human-machine
interfaces, the motivation to improve \ac{ASR} systems has increased.
Additionally, the problem still exists of aiding human understanding
in such situations, e.g. hearing aids, cochlear implants or telecommunications
systems.

One method of subspace analysis that has shown promising results is
that of \ac{NMF}. This is a relatively new method of decomposition
proposed by \citet{Lee1999}, and is notable for its parts-based decomposition.
This means it is effective in identifying and separating the parts
of a system, which has proven useful in speech enhancement. The parts
of a desired signal can be learned and later extracted from a noisy
signal. Babble filtering systems are required to be trained to recognise
the individual speaker, which is often a difficult process and a practical
limitation in these systems.

A number of different challenges have been held with the motivation
of improving the performance of \ac{ASR} systems under difficult
noise conditions \citep{Cooke2010,Barker2013,Vincent2013}. Entries
into such competitions can be broadly categorised into two categories,
those that perform recognition themselves, and those that clean the
signal and supply a cleaned signal to a standardised recogniser. Algorithms
that fall into the latter category have a possible additional application:
improving intelligibility for human listeners. It is these algorithms
that are of interest in this thesis.

There are many methods by which the successfulness of an enhancement
algorithm may be measured. However, the true performance of an enhancement
algorithm is dependent upon the application. Already mentioned have
been two broad classifications of application, those for human listening
such as hearing aids and telecommunications systems, and those for
machines, or \ac{ASR} systems. The means by which hearing is performed
is drastically different for humans and machines, and thus perceived
enhancement may also be. However, this is rarely taken into consideration.
Nor is there a standard by which enhancement should be measured across
literature, leading to difficulty in comparing different enhancement
algorithms.

\clearpage{}


\section{Research Questions}

\newcommand{\RQone}{``Are good enhancement algorithms effective for both human listeners and machine listeners? Can a generic and practical speech enhancement algorithm find application in signal enhancement and \ac{ASR}?''}
\newcommand{\RQtwo}{``What is the optimum amount of training data for state-of-the-art \ac{NMF} algorithms?''}
\newcommand{\RQthree}{``Can the results be improved by modifying algorithms to be phoneme-dependent?''}

The aim of this thesis is to address the following research questions:
\begin{enumerate}
\item \label{enu:ResQ1} \RQone{}
\item \label{enu:ResQ2}\RQtwo{}
\item \label{enu:ResQ3}\RQthree{}
\end{enumerate}

\section{Scope}

The scope of the first research area was limited to determining the
correlation between \ac{HR} and \ac{MR}. This also involved identifying
enhancement algorithms that perform as outliers, i.e., performed well
in enhancing for one category but not the other. Additionally, the
scope included providing a recommendation to future designers of enhancement
algorithms on which measures to use to efficiently classify an enhancement
algorithm's performance.

The scope of the second research area involved testing the performance
of \ac{NMF} algorithms when the amount of training data supplied
was varied. The performance measures were to meet the recommendations
of the findings of the first research question. Aims were to identify
the effects of over-training and under-training occurred, and the
number of utterances supplied for optimal training.

The scope of the third research area included development, implementation
and evaluation of phoneme-dependent modifications to existing state-of-the-art
\ac{NMF} algorithms. The evaluation measures were to meet the recommendations
of the findings of the first research question.

The scope throughout the entirety of this thesis did not encompass
specific applications, and thus did not consider special requirements
beyond normal hearing for hearing aid and cochlear implant listeners,
nor was a consumer-level implementation produced.
