
\chapter{Conclusions}

\acresetall


\section{Humanly Perceived Improvement vs. Machine Performance}

In analysing the results gathered from literature, a fair correlation
was found between \ac{PESQ} and \ac{PRR}. This indicated a correlation
between \ac{MOS} and \ac{PRR}. Therefore it was concluded that in
general and according to these results, algorithms that are successful
in \ac{HR} applications are likely to be successful in \ac{MR} applications,
and vice-versa.

Independent investigations supported the findings by directly comparing
\ac{MOS} results with \ac{PRR} results. A correlation was found
when the \ac{PRR} measure was an \ac{ASR} system's correctness.
However, a negative correlation was found when false insertions were
accounted for under the \ac{PRR}.

Furthermore, the ideal binary mask algorithm considered exhibited
a direct contrast with the general trend. The algorithm gave significant
improvements in reducing the noise, and therefore gave improvements
in the accuracy \ac{PRR}, indicating improvement for \ac{MR}. However,
the algorithm distorted the signal significantly, causing deterioration
in quality of \ac{HR}. Therefore, the correlation depended on the
requirements of the \ac{ASR} system, which are difficult to predict
at the phoneme level.

Thus it was determined that an algorithm's performance in enhancement
cannot be generalised over \ac{HR} and \ac{MR}. Although a ``good''
enhancement algorithm is likely successful in both applications, this
is not guaranteed and therefore must be verified.


\subsubsection*{Recommendations from Investigation}

It is recommended that when enhancement algorithms are analysed, as
many evaluation measures be used as possible. Many of the evaluation
measures are relatively simple to use once implemented, and so the
cost associated with increasing the range of enhancement measures
is not large in the long term. An exception is the \ac{MOS} test,
which requires a considerable level of human time. The \ac{PESQ}
score may be used as a replacement, but the correlation between \ac{PESQ}
and \ac{MOS} in not certain.

In future work, it is recommended that the correlation between \ac{PESQ}
and \ac{MOS} be investigated under the context of speech enhancement
as opposed to measuring the quality of a telecommunications medium.


\subsubsection*{Opportunity for Future Work}

There are a number of opportunities for further work to extend on
within this research area. In this thesis, machine recognition was
considered purely through one \ac{ASR} system, utilising mel-frequency
cepstral coefficients, and only considering the phoneme-level recognition.
Studies should be further conducted using alternative methods of feature
extraction for the \ac{ASR} system.

The word recognition rate, and other higher level recognition measures
should be investigated. This thesis did not take into consideration
methods \ac{ASR} systems can use to extract meaning that are robust
to errors in the phoneme-level recognition.

Additionally, with a larger test population and therefore more reliable
\ac{MOS} results better conclusions could be drawn on both the relationship
between \ac{HR} and \ac{MR} and on the relationship between \ac{PESQ}
and \ac{MOS}, both under the context of speech enhancement.

Finally, such tests should also be conducted using a more diverse
range of noise types and algorithms in order to be able to generalise
the conclusions.


\section{Assessing \acl{NMF} Algorithm Training}

The number of utterances supplied as training was found to have no
correlation with performance for the algorithms \ac{BNMF} algorithms
presented by \citet{mohammadiha2013supervised}. It was however seen
that 10 to 20 utterances should be used in order to ensure adequate
training for consistence in results.

Additionally, it was shown that in cases, under-training improved
results, indicating that a higher performance was achievable and showing
that the current training techniques were suboptimal. Thus, promise
was shown for the phoneme-dependent variations proposed to these algorithms.


\section{Phoneme-Dependent Variation Performance}

The method for randomly drawing phoneme samples and forming training
data was successfully implemented. The only noted issue in implementation
was the possibility of missing some short phonemes, such as the voiceless
stop phonemes. This had been accounted for however, by increasing
the number of samples drawn per phoneme.

Using phoneme-dependent modifications, both training and base modifications,
was not found to provide significant improvement for human listeners
against the recorded NOIZEUS noise types tested: babble, car and street.
However, slight improvements were noted against the synthetic babble.

Thus these phoneme-dependent algorithms are more effective in competing
speaker applications for a human listener. It was proposed such phoneme-dependent
algorithms may be advantageous in source-separation, where a specific
voice is required to be extracted.

Using phoneme-dependent training was found to improve the accuracy
of the mel-frequency cepstral coefficient automated speech recognition,
by reducing the number of phoneme insertions. This was especially
true for the NOIZEUS car noise recordings, but improvements were shown
in most noise types except competing speaker for machines. Phoneme-dependent
training has, therefore, been shown to be more discriminative in removing
noise for machine listeners.

Potential application for the phoneme-dependent training algorithm
lays in in-car electronics. The algorithms have been shown to improve
for car recorded noise, including muffled competing speaker voice
which may potentially come from within the car.

For such phoneme-dependent algorithms, it was found, in general, to
be advantageous to increase the number of samples per phoneme used.
No upper bound on this trend was found, although it was suspected
that one would exist.


\subsubsection*{Opportunity for Future Work}

The opportunity exists to conduct further research into using phoneme-dependence
for human listeners. It has been shown that training using utterances
is suboptimal, and additionally that phonemic training can improve
the performance for machine recognisers. A great number of variations
within the training stage of \ac{NMF} algorithms have been proposed,
that should be investigated with phoneme-dependent modifications.
