
\chapter{Conclusions}

\acresetall


\section{Humanly Perceived Improvement vs. Machine Performance}

In analysing the results gathered from literature, a strong correlation
between \ac{MOS} and \ac{PESQ} was found. Furthermore, a strong
correlation was found between \ac{PESQ} and \ac{PRR}, indicating
a correlation between \ac{MOS} and \ac{PRR}, thereby indicating
that in general, algorithms that are successful in \ac{HR} applications
are likely to be successful in \ac{MR} applications, and vice-versa.

Independent investigations supported the findings by directly comparing
\ac{MOS} results with \ac{PRR} results. A correlation was found
when the \ac{PRR} measure was an \ac{ASR} system's correctness.
However, a negative correlation was found when false insertions were
accounted for under the \ac{PRR}.

Furthermore, many situations were discovered where the general correlation
was untrue, and where an algorithm that may well be perfectly suited
to an application in ...

Thus it was determined that an algorithm's performance in enhancement
cannot be generalised. It is recommended that in future work on enhancement
algorithms be performed and analysed, that as many evaluation measures
be used as possible. Many of the evaluation measures are relatively
simple to use once implemented, an exception being \ac{MOS} which
requires a considerable level of human time.
