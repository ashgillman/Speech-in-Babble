
\chapter{Conclusions}

\acresetall


\section{Humanly Perceived Improvement vs. Machine Performance}

In analysing the results gathered from literature, a strong correlation
between \ac{MOS} and \ac{PESQ} was found. Furthermore, a strong
correlation was found between \ac{PESQ} and \ac{PRR}, indicating
a correlation between \ac{MOS} and \ac{PRR}, thereby indicating
that in general, algorithms that are successful in \ac{HR} applications
are likely to be successful in \ac{MR} applications, and vice-versa.

Independent investigations supported the findings by directly comparing
\ac{MOS} results with \ac{PRR} results. A correlation was found
when the \ac{PRR} measure was an \ac{ASR} system's correctness.
However, a negative correlation was found when false insertions were
accounted for under the \ac{PRR}.

Furthermore, the ideal binary mask algorithm considered exhibited
a direct contrast with the general trend. The algorithm gave significant
improvements in reducing the noise, and therefore gave improvements
in the accuracy \ac{PRR}, indicating improvement for \ac{MR}. However,
the algorithm distorted the signal significantly, causing deterioration
in quality of \ac{HR}.

Thus it was determined that an algorithm's performance in enhancement
cannot be generalised over \ac{HR} and \ac{MR}.


\subsubsection*{Recommendation}

It is recommended that when enhancement algorithms are analysed, as
many evaluation measures be used as possible. Many of the evaluation
measures are relatively simple to use once implemented, and so the
cost associated with increasing the range of enhancement measures
is not large in the long term. An exception is the \ac{MOS} test,
which requires a considerable level of human time. The \ac{PESQ}
score may be used as a replacement, but the correlation between \ac{PESQ}
and \ac{MOS} in not certain.

In future work, it is recommended that the correlation between \ac{PESQ}
and \ac{MOS} be investigated under the context of speech enhancement
as opposed to measuring the quality of a telecommunications medium.


\section{Assessing \acl{NMF} Algorithm Training}
