\selectlanguage{australian}%

\chapter{Methodology}

\acresetall

The algorithms outlined throughout the following chapter were implemented
in MATLAB. This environment was selected due to its availability and
ease-of-use. MATLAB includes a number of tools useful in signal processing
and statistical analysis \citep{Krauss1994,Jones1997}, and other
authors have provided tools in the MATLAB language \citep{Hoyer2004,Brookes1997,Loizou2008},
all of which may be implemented within this thesis. A limitation of
the MATLAB environment in signal processing is the comparative performance
with lower-level languages, however performance and speed were not
a primary focus in this thesis.

Data analysis was conducted using R.

Simulations were run on a MacBook Pro 64-bit system with an Intel
Core i7 2.3GHz 4- core processor with 16GB memory, running OSX 10.9.2,
MATLAB R2013a and R version 3.1.0 (2014-04-10).

Test data to was drawn from the \ac{WSJ} corpus \citep{Robinson1995}.
Different recordings were reserved for testing and training. Babble
noise was simulated by using recordings of speech, also drawn from
the \ac{WSJ} corpus.


\section{Humanly Perceived Improvement vs. Machine Performance}

The first research area sought to answer research question \vref{enu:ResQ1},
\textit{\RQone{}} The focus was to compare and contrast the performance
differences of speech enhancement algorithms when evaluated on a human
listener versus a machine listener.


\subsection{\label{sub:Method-Existing-Data}Investigating Existing Data}

Data on enhancement performance of various algorithms and under different
measures of enhancement was collected from a number of sources \citep{mohammadiha2013supervised,Wilson2008,Schmidt2006,Raj2005,Rennie2008,Weninger2011,Williamson2014,Paliwal2010,Plourde2007}
and is presented in \chapref{Apx-Data}, \secref{litresults}. As
noted in \chapref{Literature-Review} \textit{\nameref{chap:Literature-Review}},
most literature takes preference to evaluation measures indicative
of either human performance or machine performance, but not both.
Thus, methods had to be developed in order to fairly compare the the
results between different sources.

The first comparison method was \textbf{direct comparison}. This method
involved plotting human performance against machine performance, and
thus was only applicable to a small subset of the results where both
performances were measured, such as in \citep{Paliwal2010}. This
data was analysed separately for each type of algorithm investigated.
Each data set had a \ac{LOESS} model fitted, in order to quickly
and visually judge the accuracy of applying a \ac{LM} fit. An \ac{LM}
model was then applied using R's \lstinline[language=R]!lm()! linear
regression function. The main limitation of the direct comparison
method was the limited dataset available.

A \textbf{grouped comparison} was also conducted. This was performed
as above, however similar algorithms were grouped into algorithm classes.

The second comparison method..

A limitation in this analysis is that the test conditions are not
fair: different noise types, subtle differences in algorithms BLURGH


\subsection{Independent Investigation}

Due to the lack of existing data on which conclusions can be drawn
on the human vs. machine success of various speech enhancement algorithms,
independent tests were required to be run. In order to answer the
research question, a number of enhancement algorithms were required
to be used to enhance speech, and a number of evaluation measures,
of both human perceptual and machine recognition types, were to be
used to measure the performance.

The Algorithms to be used were:
\begin{itemize}
\item Supervised \ac{HMM} with \ac{BNMF} output distributions \citep{mohammadiha2013supervised}.
\item Online \ac{HMM} with \ac{BNMF} output distributions \citep{mohammadiha2013supervised}.
\end{itemize}
The performance measures to be used are \ac{PESQ} \citep{InternationalTelecommunicationUnion2001},
\ac{MOS}, \ac{CCR} \citep{InternationalTelecommunicationUnion1996}
and \ac{PRR}.

\ac{PESQ} was selected as a performance measure due to its extensive
usage in literature. Using the same evaluation measure allowed a certain
degree of comparability with others' results. The \ac{PESQ} implementation
used was a MATLAB implementation provided by \citet{Loizou2008}.

\ac{MOS} and \ac{CCR} were selected due to their subjective qualities.
Although subjective measures are generally avoided, they were decided
to be useful in the experimental context, as human perception is itself
subjective. \ac{MOS}, like \ac{PESQ}, had been used throughout literature.
\ac{CCR} was included also since the scale used lent itself to measuring
enhancement, and is in many ways more similar to \ac{PESQ} due to
the fact it measures improvement or degradation from a reference.

A MATLAB implementation of the \ac{MOS} test exists \citep{Ruzanski2009},
however the implementation was not flexible and the script was not
found to be suited to the required test. A new \ac{MOS} and \ac{CCR}
script was created, in alignment with the requirements outlined by
the \citet{InternationalTelecommunicationUnion1996}.

The scales used are given in \tabref{MOS-CCR-Scales}. Two \ac{MOS}
scales were used, one measuring quality opinion, and one measuring
required effort opinion.

\begin{table}[h]
\protect\caption{\label{tab:MOS-CCR-Scales}Scales used in subjective tests}


\subfloat[\acs{MOS} listening quality scale]{

\begin{centering}
\begin{tabular}{|c|c|}
\hline 
Quality of the speech & Score\tabularnewline
\hline 
\hline 
Excellent & 5\tabularnewline
\hline 
Good & 4\tabularnewline
\hline 
Fair & 3\tabularnewline
\hline 
Poor & 2\tabularnewline
\hline 
Bad & 1\tabularnewline
\hline 
\end{tabular}
\par\end{centering}

\selectlanguage{english}%
\selectlanguage{english}%
}

\subfloat[\acs{MOS} listening effort scale]{

\centering{}%
\begin{tabular}{|c|c|}
\hline 
Effort required to understand the meanings of sentences & Score\tabularnewline
\hline 
\hline 
Complete relaxation possible; no effort required & 5\tabularnewline
\hline 
Attention necessary; no appreciable effort required & 4\tabularnewline
\hline 
Moderate effort required & 3\tabularnewline
\hline 
Considerable effort required & 2\tabularnewline
\hline 
No meaning understood with any feasible effort & 1\tabularnewline
\hline 
\end{tabular}}

\subfloat[\acs{CCR} scale]{

\begin{centering}
\begin{tabular}{|c|c|}
\hline 
The quality of the second compared to the quality of the first & Score\tabularnewline
\hline 
\hline 
Much Better & 3\tabularnewline
\hline 
Better & 2\tabularnewline
\hline 
Slightly Better & 1\tabularnewline
\hline 
About the Same & 0\tabularnewline
\hline 
Slightly Worse & -1\tabularnewline
\hline 
Worse & -2\tabularnewline
\hline 
Much Worse & -3\tabularnewline
\hline 
\end{tabular}
\par\end{centering}

\selectlanguage{english}%
\selectlanguage{english}%
}
\end{table}


When performing the \ac{MOS} and \ac{CCR} tests, some of the recommendations
could not be met. Sound-proof rooms were not available, as recommended
in \citep{InternationalTelecommunicationUnion1996} Annex A, and so
environmental, room and vehicle noise were not able to be tightly
controlled. This restriction is important when testing the quality
of telecommunications systems as stationary as well as non-stationary
noise needs to be tested. However, in the current experiments, non-stationary
noise is of interest. Therefore it is important that non-stationary
noise be removed, such as traffic, however some stationary noise within
the testing environment is acceptable. These operating conditions
were achievable.

Both \ac{HTK} and Sphinx were both considered for the \ac{ASR} system
for the \ac{PRR} results. Both offer comparable performance in recognition
and processing performance \citep{Vertanen2006}. \ac{HTK} was 


\section{Assessing \acl{NMF} Algorithm Training}

The second research area sought to answer research question \ref{enu:ResQ2},
\textit{\RQtwo{}} The focus was to investigate the training requirements
of supervised speech enhancement algorithms, and to investigate means
of improving the quality of training.


\subsection{\label{sub:Investigating-Training-Req}Investigating Training Requirements}

An experiment was conducted, investigating the effects of the performance
of enhancement. Firstly, test data was generated, using the code given
in \lstref{createTestData}, \textit{\nameref{lst:createTestData}}.
The data was formed as \lstinline[language=bash]!.wav! files consisting
of:
\begin{itemize}
\item training files, for use by supervised enhancement algorithms, for
both \ac{SoI} and noise recordings;
\item test files, or dirty files, for the enhancement algorithms to operate
on;
\item the clean file, for the evaluation algorithms to compare the enhancement
with; and
\item the clean noise file.
\end{itemize}
A range of training files were created, varying the number of utterances
included. The \ac{WSJ} corpus included 90 utterances per speaker%
\footnote{There are 90 utterances for the speakers reserved for training, \lstinline[language=bash]!si_dt!,
which were used in this test.%
} \citep{Fransen1994}, so 10 utterances were used for the test speech,
and the number of utterances used for training were varied between
1, 3, 5, 10, 15, 20, 30, 40, 50, 60, 70 and 80. The specific 10 utterances
used for testing were banned from being used for training to prevent
bias results.

Similarly, a range of test files were created by mixing a \ac{SoI}
recording with noise, varying the \ac{SNR}. The \ac{SNR} was varied
between -6dB, -3dB, 0dB, 3dB and 6dB. Additionally, a number of other
prerecorded noise types were drawn from the NOIZEUS corpus presented
in \citep{Hu2006}. The speakers used in each test are outlined in
\tabref{test-params}.

\begin{table}
\protect\caption{\label{tab:test-params}Simulations and associated parameters}


\begin{centering}
\begin{tabular}{|c|c|c|}
\hline 
Test Number & \ac{SoI} (sex) & Noise\tabularnewline
\hline 
\hline 
1 & c3c (f) & Competing speaker: c3f (m)\tabularnewline
\hline 
2 & c3l (f) & Competing speaker: c31 (m)\tabularnewline
\hline 
3 & c3s (m) & Competing speaker: c31 (m)\tabularnewline
\hline 
4 & c3s (m) & Synthetic babble: c31 (m) + c34 (m) + c35 (m)\tabularnewline
\hline 
5 & c3s (m) & Synthetic babble: c3f (m) + c3c (f) + c35(m)\tabularnewline
\hline 
6 & c3s (m) & Synthetic babble: c3f (m) + c3c (f)\tabularnewline
\hline 
7 & c3s (m) & Competing speaker: c3f(m)\tabularnewline
\hline 
8 & c3s (m) & NOIZEUS car\tabularnewline
\hline 
9 & c3s (m) & NOIZEUS babble\tabularnewline
\hline 
10 & c3s (m) & NOIZEUS street\tabularnewline
\hline 
\end{tabular}
\par\end{centering}

\selectlanguage{english}%
\selectlanguage{english}%
\end{table}


The enhancement algorithms used were those proposed by \citet{mohammadiha2013supervised},
two \ac{HMM} algorithms with \ac{BNMF} output distributions. One
uses a fully supervised method, and the other does not train for noise,
but estimates it online. The enhancement was conducted under all combinations
of utterances and \acp{SNR}. The code used to conduct the test is
given in \lstref{varyingTrainingTest}, \textit{\nameref{lst:varyingTrainingTest}}.

The performance of the enhancement algorithm was then analysed using
the \ac{PESQ} score, the \ac{PESQ} score improvement compared with
the dirty mixed signal, the segmental \ac{SNR}, and the segmental
\ac{SNR} improvement. The code used to perform this analysis is given
in \lstref{varyingTrainingAnalysis}, \textit{\nameref{lst:varyingTrainingAnalysis}}.\selectlanguage{english}%

