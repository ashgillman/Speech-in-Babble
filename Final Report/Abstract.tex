\begin{abstract}
Speech enhancement is an important field in modern electronics, finding
many applications from hearing aids to enhancing audio/visual recordings
and automated speech recognisers in commercial products. These can
range from mobile phones to modern televisions and automobile electronics.
A variety of techniques are available to measure speech enhancement.
These fall into three categories: (1) those for measuring speech enhancement
for a human recogniser; (2) those for measuring speech enhancement
for a machine recogniser; and (3) those measuring speech enhancement
using purely statistical methods. However, when an enhancement algorithm
is proposed it is often measured by techniques from only one of those
categories. Due to there being fundamental differences from a human
listener to a machine listener, it is hypothesised that the use of
a variety of enhancement measures is necessary to properly measure
the performance of an enhancement algorithm. Results indicated that
the hypothesis was true, and that it is possible for a speech enhancement
method that performs well for a machine recogniser to perform poorly
for a human recogniser, and vice-versa. Therefore, it is important
to encapsulate a variety of measures in testing. The recommended tests
include PESQ, ASR performance, statistical measures such as segmental
SNR and where possible, MOS.

Investigations were also performed into the effects of varying the
number of sentences used for training. It was found that there was
no correlation, and that varying training did not effect the average
results, however it did effect the variance and therefore reliability
of results, if undertrained. Adequate training was found to be 10
to 20 utterances. Additionally, it was found that training using utterances
was suboptimal. This showed promise for the third research area, which
investigated an alternative training to utterances.

The proposed algorithm training modification was a phoneme-dependent
modification to existing non-negative matrix factorisation algorithms.
This algorithm was found to provide better accuracy for machine recognisers,
especially so under in-car noise conditions. The changes did not improve
human intelligibility on average, but did perform more consistently,
being less likely to distort speech. The system showed capability
of performing well for in-car electronics, such as for voice-controlled
navigational electronics or car radios.\end{abstract}

