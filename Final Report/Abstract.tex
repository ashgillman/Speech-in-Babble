\begin{abstract}
Speech enhancement is an important field in modern electronics, finding
many applications from hearing aids to enhancing audio/visual recordings
and automated speech recognisers in commercial products. These can
range from mobile phones to modern televisions and automobile electronics.
A variety of techniques are available to measure speech enhancement.
These fall into three categories: (1) those for measuring speech enhancement
for a human recogniser; (2) those for measuring speech enhancement
for a machine recogniser; and (3) those measuring speech enhancement
using purely statistical methods. However, when an enhancement algorithm
is proposed it is often measured by techniques from only one of those
categories. Due to there being fundamental differences from a human
listener to a machine listener, it is hypothesised that the use of
a variety of enhancement measures is necessary to properly measure
the performance of an enhancement algorithm. Results indicated that
the hypothesis was true, and that it is possible for a speech enhancement
method that performs well for a machine recogniser to not perform
well for a human recogniser, and vice-versa. Therefore, it is important
to encapsulate a variety of measures in testing. The recommended tests
include PESQ, ASR performance, statistical measures such as segmental
SNR and where possible, MOS.

In addition, a phoneme dependent modification to existing non-negative
matrix factorisation algorithms was proposed and implemented. This
algorithm was found to provide better accuracy for machine recognisers.
The changes did not improve human intelligibility on average, but
did perform more consistently, being less likely to distort speech.\end{abstract}

